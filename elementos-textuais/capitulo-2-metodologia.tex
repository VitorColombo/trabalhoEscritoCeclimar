\chapter{Metodologia}\label{metodologia}
% ---
Para a realização do embasamento deste trabalho, foram utilizadas tanto a metodologia de pesquisa bibliográfica quanto a de pesquisa documental. A primeira foi essencial para o levantamento de metodologias já consolidadas e amplamente estudadas, como as que serão abordadas no referencial teórico e, a seguir, nesta seção. Já a segunda foi empregada para identificar diferentes aplicações correlatas e para a elaboração do referencial teórico, além de ter sido utilizada na análise de dados internos do CECLIMAR, conforme comentado na conforme comentado na \hyperref[chapter:intro]{introdução} deste trabalho.

Segundo \citeonline{gil2002elaborar}, a pesquisa bibliográfica é desenvolvida com base em materiais como livros e artigos científicos, ou seja, materiais já consolidados. Trata-se de uma pesquisa de grande importância, pois permite que os pesquisadores acessem diversos dados e informações dispersos que, individualmente, seriam muito trabalhosos e custosos de se coletar. Nesse tipo de pesquisa, entretanto, é necessário ter cuidado com citações de terceiros, que podem interpretar de forma equivocada algum dado ou informação originalmente levantados.

Ainda segundo o autor, a pesquisa documental se diferencia pela natureza das fontes de informação. Enquanto as pesquisas bibliográficas consistem essencialmente em um apanhado de contribuições de diversos autores sobre determinado assunto, a pesquisa documental ocorre por meio de materiais que ainda não receberam tratamento analítico ou que podem ser reelaborados, a depender dos objetos de pesquisa. As fontes da pesquisa documental são mais diversas e podem incluir conversas pessoais, entrevistas, documentos ou sites.

A metodologia de pesquisa bibliográfica foi realizada através da plataforma Google Scholar, publicações presentes no portal do Sistema de Informação sobre a Biodiversidade Brasileira (SiBBr), além de livros disponibilizados na biblioteca do Instituto Federal Campus Osório. Já a metodologia de pesquisa documental foi levantada a partir de arquivos internos, relatórios das Nações Unidas e conteúdos disponibilizados por desenvolvedores ou organizações que participaram do desenvolvimento das aplicações correlatas.

Após a fase de revisão bibliográfica se iniciará o desenvolvimento do sistema. Esta fase será realizada a partir do levantamento de requisitos junto de profissionais do CECLIMAR e, os requisitos levantados serão cadastrados e refinados para desenvolvimento cíclico do sistema. Cada ciclo visará a entrega de um produto com incrementos de requisitos pré estabelecidos. Ao final de cada um dos ciclos de desenvolvimento o sistema será disponibilizado para testes com servidores do CECLIMAR, os \textit{feedbacks} recebidos serão analisados, refinados e postos para desenvolvimento no ciclo seguinte. Um ponto de atenção no desenvolvimento desse sistema é que, por se tratar de um sistema de ciência cidadã, deve estar adequado com a LGPD para que possa ser publicado na Play Store para o uso da sociedade.

% ---
\section{Metodologia de desenvolvimento de \textit{software}}
% ---

Para o desenvolvimento deste projeto foi escolhida uma abordagem de metodologia ágil relacionada ao ciclo iterativo incremental. Utilizando o Kanban como método de gestão de fluxo de trabalho, a fim de melhorar a eficiência e qualidade do produto final a partir da visualização das tarefas. A ferramenta escolhida para realizar este gerenciamento foi o Jira.

Segundo \citeonline{pressman2011engenharia}, os ciclos de desenvolvimento incremental podem ser divididos em 5 principais etapas: comunicação, planejamento, modelagem (análise e projeto), construção (codificação e testes) e emprego (entrega, \textit{feedback}). As etapas por ele descritas serão aplicadas neste projeto.

A comunicação será marcada por reuniões agendadas com os profissionais do CECLIMAR para definição de escopo e levantamento de requisitos do sistema. O planejamento será a análise, o refinamento e as definições de quais funcionalidades serão desenvolvidas em cada ciclo. Durante a modelagem será realizada a prototipação e análise dos pontos levantados na etapa anterior. Na construção será o momento onde se dará a codificação e os testes da aplicação. E, por fim, durante o emprego o sistema será disponibilizado para os profissionais do CECLIMAR e uma porcentagem de usuários que realizarão testes e retornarão \textit{feedback} que serão analisados, catalogados e inseridos no \textit{Backlog} para serem puxados em um ciclo posterior.

O uso desta metodologia tem o objetivo de realizar uma primeira entrega que possa ser considerada um Mínimo Produto Viável (MPV) atendendo aos requisitos básicos propostos inicialmente, mesmo que ainda se note a ausência de funcionalidades complementares. Esse MVP se tornará então a base de avaliação que permitirá identificar necessidades adicionais e ajustes necessários. Com base nessa análise, é planejado o próximo incremento, ajustando a primeira entrega e adicionando novas funcionalidades conforme as necessidades.

Neste projeto foi montado um fluxo de trabalho no Jira para desenvolvimento de \textit{software} visando auxiliar no processo e manter a visibilidade das tarefas de ponta a ponta. Para o workflow principal, foi montado um esquema com \textit{Backlog}, Refinamento, Em desenvolvimento, Aguardando teste, Em teste, Correção de \textit{bugs} e \textit{Done} (Figura~\ref{fig:fluxoJira}).

\begin{figure}[htb]
    \centering
    \includegraphics[width=0.8\textwidth]{imagens/fluxoJira.png}
    \caption{Fluxo de trabalho do Jira gerado para o desenvolvimento do projeto.}
    \legend{Fonte: Autor}
    \label{fig:fluxoJira}
\end{figure} 

O \textit{Backlog} é a coluna onde todas as tarefas, user stories e \textit{bugs} serão inicialmente posicionados. Nesta etapa, as tarefas serão priorizadas antes de andarem para o próximo estágio.

No Refinamento, as tarefas puxadas do \textit{Backlog} são detalhadas para um desenvolvimento mais assertivo. São refinados critérios de aceitação, estimativas de tempo e algum débito técnico.

As tarefas que estiverem em desenvolvimento são as que tiveram, efetivamente, o seu desenvolvimento iniciado. Assim que o desenvolvimento estiver finalizado, as tarefas serão transferidas para aguardando teste, onde ficarão até serem puxadas para testes mais detalhados.

No estágio de teste, os critérios de aceite e a presença de \textit{bugs} serão testados com o intuito de manter a qualidade do produto final. As tarefas que tiverem \textit{bugs} ou divergências de regras de negócio identificadas serão movidas para a coluna Correção de \textit{bugs} para que sejam corrigidas.

E, por último, após a homologação das tarefas nas etapas anteriores as tarefas são movidas para \textit{Done} que indicará sua finalização.


% ----------------------------------------------------------
\chapter{Introdução} \label{chapter:intro}
% ----------------------------------------------------------

Este projeto de conclusão de curso é fruto de uma colaboração com o CECLIMAR (Centro de Estudos Costeiros 
Limnológicos e Marinhos), onde se identificou a necessidade de aprimorar o processo de coleta e monitoramento 
de dados da fauna costeira. Diante dos desafios atuais, propõe-se o desenvolvimento de um aplicativo de 
Ciência Cidadã \cite{Martins_Cabral_2021}. Este tipo de aplicativo permite que o público geral contribua 
com dados científicos, aumentando o alcance e a eficiência da pesquisa.
Atualmente, no CECLIMAR, a coleta dos dados de monitoramento é realizada manualmente: as pessoas enviam 
informações via \citeonline{whatsApp}, e um pesquisador do órgão encaminha os dados principais para uma 
bolsista, que os classifica e registra em uma planilha eletrônica, armazenando as fotos em uma pasta do 
\citeonline{googleDrive}. Este método manual tem levado a inconsistências nos dados e exigido revisões 
periódicas pelo gestor do projeto.

Diante desse cenário, a solução proposta é a criação de uma aplicação que facilite o registro e a avaliação das ocorrências  
do objeto de estudo, trazendo também mais visibilidade e proximidade com população que contribui com o projeto. 
A classificação taxonômica dos animais será feita por meio do 
sistema, indicando características de cada espécie bem como o estado de decomposição de cada registro. 
A automação desses processos visa reduzir as inconsistências e otimizar a gestão dos dados coletados.
Este projeto tem como objetivo geral desenvolver um aplicativo de Ciência Cidadã para otimizar o processo 
de coleta, classificação e gestão de dados da fauna costeira voltado para atender as demandas de 
profissionais do CECLIMAR.

A aplicação possui como objetivos específicos automatizar o processo de coleta e armazenamento das 
ocorrências para garantir precisão dos dados e facilitar os registros de observações da fauna costeira 
com a ajuda da população a partir de uma interface amigável e intuitiva. Padronizar os registros 
para garantir um banco de dados robusto, reduzindo inconsistências e minimizando a necessidade de 
revisões periódicas, e facilitar a realização de pesquisas e a 
análise de dados recebidos, liberando recursos para outras atividades de pesquisa. Além de promover 
a participação ativa da comunidade na conservação da biodiversidade costeira e no monitoramento ambiental.

Com este contexto, podemos afirmar que este trabalho possui seu desenvolvimento alinhado com a Agenda 2030 da ONU \cite{onu2015agenda2030}, usando a integração e aplicação de tecnologias no desenvolvimento sustentável, visando abranger os itens 14 (Vida na água), 15 (Vida terrestre) e 9 (Indústria, inovação e infraestrutura). Além disso, o projeto busca promover a difusão e aplicação dos princípios da ciência cidadã, ao facilitar a colaboração entre a comunidade e cientistas. A ciência cidadã amplia a participação pública na pesquisa científica, proporcionando uma abordagem colaborativa e inclusiva na gestão ambiental. Portanto, este trabalho busca não apenas oferecer soluções práticas para o monitoramento da fauna na região costeira do Rio Grande do Sul, mas também promover uma mudança de paradigma na forma como a ciência é realizada, enfatizando a importância da participação e colaboração da comunidade na construção de um futuro sustentável. 

\chapter{Cronograma}\label{cronograma}

O cronograma ilustrado na Tabela ~\ref{tab:cronograma} representa a trajetória de 
desenvolvimento do projeto ao longo de 16 meses, bem como as divergências e 
replanejamentos que foram realizados no decorrer do tempo. O trabalho teve seu 
início no dia 28 de março de 2024, com a realização de uma reunião de definição 
de escopo junto do orientador Marcelo Paravisi. No dia 1º de abril, deu-se 
continuidade com a definição do tema, em uma reunião externa com um membro do 
CECLIMAR. Essa etapa foi fundamental para delimitar o foco do trabalho e aprofundar 
a compreensão das dificuldades, necessidades, oportunidades e pontos críticos do projeto

Os meses seguintes, de maio a julho de 2024, foram dedicados à revisão bibliográfica 
e à definição metodológica. A definição metodológica se deu durante o mês de maio 
tendo início no dia 06, enquanto a parte de revisão bibliográfica se iniciou no 
dia 05 do mesmo mês e foi finalizada no final de julho. 

O levantamento de requisitos e regras de negócio se deu a partir do dia 10 de 
maio e se estendeu até o final de agosto para que os alinhamentos de definições 
com a equipe do CECLIMAR fossem mais abertos e constantes. É importante ressaltar 
que durante o desenvolvimento houve mudanças de funcionalidades e ajustes de regras 
de negócio durante o período de testes que fizeram com que fosse necessário 
revisitar esse tópico. 

Esse período inicial de estruturação se mostrou essencial para assegurar uma base 
sólida para iniciar o desenvolvimento. A parte de desenvolvimento técnica se 
iniciou com a organização da implementação do sistema no dia 20 de maio de 2024 
e, com base nisso, se iniciou a codificação do aplicativo no dia 1º de junho do mesmo ano. 

Inicialmente, o desenvolvimento tinha um planejamento de conclusão para o final 
de outubro, porém, embora as atividades tenham progredido conforme o cronograma, 
em uma reunião com o orientador do projeto foi decido realizar uma alteração no 
cronograma adicionando uma etapa de testes disponibilizando a aplicação com 
testadores tanto do projeto parceiro, como terceiros que contribuíram para que 
a aplicação adquirisse uma maturidade maior. Com isso, o prazo final de codificação 
foi replanejado para o início de maio de 2025 (Tabela ~\ref{tab:cronograma}).

A parte de redação da parte escrita do trabalho de conclusão foi iniciada 
paralelamente com a codificação para obter uma documentação mais precisa 
do processo e estava planejada para ser concluída até o final de novembro 
de 2024, porém, também foi afetada pelo replanejamento. Em vermelho na 
Tabela ~\ref{tab:cronograma} podemos ver que esta etapa foi realizada dentro 
do prazo previsto até o mes de outubro, porém em novembro foi adiada para maio 
e junho para priorizar a conclusão da codificação e o aprimoramento da qualidade 
da aplicação.
O desenvolvimento continuou sendo a atividade central até maio de 2025, acompanhado 
de execuções constantes de testes. No mês de maio de 2025 a redação do trabalho 
retornou e se estendeu, junto da revisão textual, até o final de junho para ser 
entregue dentro da data máxima de 26 de junho. A apresentação para a banca até o 
momento havia sido definida, porém tem prazo máximo de 11 de julho de 2025.

Esse cronograma evidencia o fluxo de trabalho realizado desde o início do 
planejamento do projeto, bem como as alterações ocorridas durante seu desenvolvimento. 
Para a elaboração desse material, foi fundamental que os períodos de escrita e codificação 
estivessem bem alinhados, permitindo traçar e documentar, com maior precisão, a linha do 
tempo apresentada, desde o início até a entrega final planejada.

% a tabela realmente ficou bem representativa? 
\begin{table}[ht]
\centering
\renewcommand{\arraystretch}{1.5}
\setlength{\tabcolsep}{4pt} 
\resizebox{\textwidth}{!}{
\begin{tabular}{|>{\bfseries}l|*{11}{>{\centering\arraybackslash}p{1.6cm}|}}
\hline
\rowcolor{gray!20}
Meses & \rotatebox{90}{Reunião def. de escopo} & \rotatebox{90}{Definição de tema} & \rotatebox{90}{Revisão bibliográfica} & \rotatebox{90}{Def. metodológica} & \rotatebox{90}{Levantamento de requisitos  } & \rotatebox{90}{Organização de implementação} & \rotatebox{90}{Desenvolvimento} & \rotatebox{90}{Redação de TCC} & \rotatebox{90}{Revisão textual} & \rotatebox{90}{Apresentação} & \rotatebox{90}{Testes} \\ \hline
Mar/24 & \cellcolor{gray!30} & & & & & & & & & & \\ \hline
Abr/24 & & \cellcolor{gray!30} & & & & & & & & & \\ \hline
Mai/24 & & & \cellcolor{gray!30} & \cellcolor{gray!30} & \cellcolor{gray!30} & \cellcolor{gray!30} & & & & & \\ \hline
Jun/24 & & & \cellcolor{gray!30} & & \cellcolor{gray!30} & \cellcolor{gray!30} & \cellcolor{gray!30} & \cellcolor{gray!30} & & & \\ \hline
Jul/24 & & & \cellcolor{gray!30} & & \cellcolor{gray!30} & & \cellcolor{gray!30} & \cellcolor{gray!30} & & & \\ \hline
Ago/24 & & & & & \cellcolor{gray!30} & & \cellcolor{gray!30} & \cellcolor{gray!30} & & & \\ \hline
Set/24 & & & & & & & \cellcolor{gray!30} & \cellcolor{gray!30} & & & \\ \hline
Out/24 & & & & & & & \cellcolor{gray!30} & \cellcolor{gray!30} & & & \\ \hline
Nov/24 & & & & & & & \cellcolor{blue!30} & \cellcolor{red!30} & \cellcolor{red!30} & & \\ \hline
Dez/24 & & & & & & & \cellcolor{blue!30} & & & \cellcolor{red!30} & \cellcolor{blue!30} \\ \hline
Jan/25 & & & & & & & \cellcolor{blue!30} & & & & \cellcolor{blue!30} \\ \hline
Fev/25 & & & & & & & \cellcolor{blue!30} & & & & \cellcolor{blue!30} \\ \hline
Mar/25 & & & & & & & \cellcolor{blue!30} & & & & \cellcolor{blue!30} \\ \hline
Abr/25 & & & & & & & \cellcolor{blue!30} & & & & \cellcolor{blue!30} \\ \hline
Mai/25 & & & & & & & \cellcolor{blue!30} & \cellcolor{blue!30} & & & \cellcolor{blue!30} \\ \hline
Jun/25 & & & & & & & & \cellcolor{blue!30} & \cellcolor{blue!30} & \cellcolor{blue!30} & \cellcolor{blue!30} \\ \hline
\end{tabular}%
}
\caption{Cronograma de desenvolvimento — Cinza: Planejamento inicial, Azul: Replanejamento, Vermelho: Prazos adiados.}
\label{tab:cronograma}
\legend{Fonte: Autor}
\end{table}